
\documentclass[a4paper,11pt,dvidpdfmx,uplatex]{jsreport}

\usepackage[dvipdfmx]{graphicx}

\title{Search for third-generation leptquarks at $\sqrt{s}=13$ TeV with ATLAS detector}
\author{Kosuke Takeda}
\date{XXXX 2021}

\begin{document}

\maketitle

\begin{abstract}
  素粒子標準模型を「感覚的に」理解することを目的としたメモ兼ドキュメント。
  最終的には
  。。。
  を理解することを目的としている。
  どこから読んでも良いように、随所に非常に重複する形で脚注による補足を設けた。
  はじめから最後まで読む人には「何回脚注で説明するねん」と思われるかもしれないが、途中からつまみ読みする人には初めての脚注だったりする。
  どちらかといえば途中からつまみ読みする人に向けたドキュメントでもあるので、辛抱頂きたい。
\end{abstract}


\chapter{導入}
\section{ローレンツ変換}
「ある慣性系$S$からある慣性系$S\prime$へ座標変換したときに、物理法則は変わらない」というのが相対論からの要請(相対論的不変性)である。
この座標変換をローレンツ変換といい、
\begin{equation}
  \Lambda_{\mu}^{\nu} = \left(
    \begin{array}{cccc}
      \gamma       & -\beta\gamma & 0 & 0 \\
      -\beta\gamma & \gamma       & 0 & 0 \\
      0            & 0            & 1 & 0 \\
      0            & 0            & 0 & 1
    \end{array}
  \right)
\end{equation}
で表される行列を用いると
\begin{equation}
  x\prime^{\mu} = \Lambda^\mu_\nu x^\nu
\end{equation}


\chapter{量子力学}
\section{基本的な関係式一覧}

\subsubsection{転置}
\begin{equation}
  (H_{ij})^{T}=(H_{ji})
\end{equation}
対角成分は変換を受けず、非対角成分のみが変換を受ける。

\begin{equation}
  \left(
    \begin{array}{cccc}
      a_{11} & a_{12} & a_{13} & a_{14} \\
      a_{21} & a_{22} & a_{23} & a_{24} \\
      a_{31} & a_{32} & a_{33} & a_{34} \\
      a_{41} & a_{42} & a_{43} & a_{44} 
    \end{array}
  \right)
    ^{T}
    =
  \left(
    \begin{array}{cccc}
      a_{11} & a_{21} & a_{31} & a_{41} \\
      a_{12} & a_{22} & a_{32} & a_{42} \\
      a_{13} & a_{32} & a_{33} & a_{43} \\
      a_{14} & a_{24} & a_{34} & a_{44} 
    \end{array}
  \right)
\end{equation}

\subsubsection{エルミート共役}
「エルミート共役をとる」とは、(1)行列を転置して、(2)複素共役をとったもの。
\begin{equation}
  (H_{ij})^{\dagger}=(H_{ji})^{*}
\end{equation}


\subsubsection{エルミート演算子、エルミート行列}
エルミート共役をとった演算子が自分自身である演算子のこと。
「この行列はエルミートである」とも表現される
\begin{equation}
H^{\dagger}=H
\end{equation}

\subsubsection{ユニタリー演算子、ユニタリー行列}
エルミート共役をとったものと自分自身の積が可換で、かつ単位行列に等しい演算子。
\begin{equation}
H^{\dagger}H=HH^{\dagger}=I
\end{equation}

\subsubsection{パウリ行列}
\begin{equation}
  \sigma^1 = 
  \left(
    \begin{array}{cc}
      0 & 1 \\
      1 & 0 
    \end{array}
  \right),~
  \sigma^2 = 
  \left(
    \begin{array}{cc}
      0 & -i \\
      i & 0 
    \end{array}
  \right),~
  \sigma^3 = 
  \left(
    \begin{array}{cc}
      1 & 0 \\
      0 & -1 
    \end{array}
  \right)
\end{equation}

\section{基本的な関係式一覧(量子力学)}
\begin{eqnarray}
  E &\to& i\frac{\partial}{\partial t} \\
  p &\to& -i\nabla \\
  p^{\mu} &\to& i\partial^{\mu} = i(\partial^0, -\partial^1, -\partial^2, -\partial^3)
\end{eqnarray}





\input{chapter/group_theory}


\chapter{クラインゴルドン方程式}
\section{クラインゴルドン方程式}



\chapter{ディラック方程式}
\section{ディラック方程式}

自由\footnote{ゲージ場と相互作用していないことを意味する。}粒子の従うディラック方程式は
\begin{equation}
  (i\gamma^{\mu}\partial_{\mu}-mI_4)\psi(x)=0
\end{equation}
ここで、$m$は粒子の質量、$\psi$は粒子のスピノルである。
$\gamma$行列は$4\times4$成分持っている行列であり、
\begin{equation}
  \{ \gamma^{\mu}, \gamma^{\nu} \} = 2\eta^{\mu\nu}I
\end{equation}
の反交換関係を満たしていればどのような表式でもよい\footnote{具体的な表式は無限通り存在する。}。

このディラック方程式を導くラグランジアンは
\begin{equation}
  \mathcal{L}=\bar{\psi}(i\gamma^{\mu}\partial_{\mu}-m)\psi
\end{equation}

ディラック粒子の例として電子を考えると、電磁場と相互作用するディラック方程式は
\begin{equation}
  (i\gamma^{\mu}(\partial_{\mu}-iqA_\mu)-m)\psi(x)=0
\end{equation}

\section{双一次形式}

次の量をスピノル場$\psi(x)$の双一次形式と呼ぶ。
\begin{equation}
  \bar{\psi(x)}\Gamma\psi(x)
\end{equation}
ここで、$\bar{\psi(x)}$は$\psi(x)$のディラック共役と呼ばれ、次式で定義される。
\begin{equation}
  \bar{\psi(x)} \equiv \psi^{\dagger}(x)\gamma^0
\end{equation}
$\Gamma$は$\gamma$行列の積で与えられる量で、特に
\begin{equation}
  \Gamma=I_4,~~\gamma^\mu,~~ \sigma^{\mu\nu}=\frac{i}{2}[ \gamma^\mu,\gamma^\nu ],~~\gamma^5,~~ \gamma^\mu\gamma^5
\end{equation}
が重要である。

\begin{table}[]
  \centering
  \begin{tabular}{ccc}
    名称         & 双一次形式 & パリティ変換後 \\
    スカラー     & $\bar{\psi}\psi$                    & $\bar{\psi}\psi$   \\
    擬スカラー   & $i\bar{\psi}\gamma^5\psi*$          & $-i\bar{\psi}\gamma^5\psi*$         \\
    ベクトル     & $\bar{\psi}\gamma^{\mu}\psi$        & $\bar{\psi}\gamma_{\mu}\psi$    \\
    軸性ベクトル & $\bar{\psi}\gamma^5\gamma^\mu\psi$  & $-\bar{\psi}\gamma^5\gamma_\mu\psi$ \\
    テンソル     & $\bar{\psi}\sigma^{\mu\nu}\psi$     & $\bar{\psi}\sigma_{\mu\nu}\psi$   
  \end{tabular}
\end{table}

\section{カレント}

\section{カイラルスピノル}

$\gamma^{\mu}$行列の具体的な表式は無限個存在している。
どのような表示形式を選んでも本質は変わらないので、注目したい性質に合わせて選べば良い\footnote{物の性質は変わらず何で観測するか・調べるかで見え方が変わっているだけだと感じる。}。
ここではカイラル表示(ワイル表示)の$\gamma^{\mu}$行列について調べてみる。

\begin{equation}
  \gamma_W^0 \equiv
  \left(
    \begin{array}{cc}
      0   & I_2 \\
      I_2 & 0 
    \end{array}
  \right)
    ,~
  \gamma_W^j \equiv
  \left(
    \begin{array}{cc}
      0        & -\sigma^j \\
      \sigma^j & 0 
    \end{array}
  \right)
\end{equation}
上記で定義された表現方式をカイラル表示(ワイル表示)と呼び、以下の議論で見るようにカイラリティに関して重要な表式になる。
この表式を用いると、$\gamma^5$は
\begin{eqnarray}
  \gamma^5 = i\gamma^0\gamma^1\gamma^2\gamma^3 = ... &=& 
  i 
  \left(
    \begin{array}{cc}
      -\sigma^1\sigma^2\sigma3  & 0 \\
      0                         & \sigma^1\sigma^2\sigma3 
    \end{array}
  \right)\\
  &=& 
  \left(
    \begin{array}{cc}
      I_2 & 0 \\
      0   & -I_2
    \end{array}
  \right)
\end{eqnarray}
となる。
カイラル表示の特徴は$\gamma^5$が対角化された表式\footnote{対角成分しか値を持っていない行列のこと。それ以外は非対角成分と呼ぶ}になっているという点である。
この$\gamma^5$を4成分スピノルに作用させると、
\begin{eqnarray}
  \gamma^5\psi &=& 
  \left(
    \begin{array}{cc}
      I_2 & 0 \\
      0   & -I_2
    \end{array}
  \right)
  \left(
    \begin{array}{c}
      \xi(x) \\
      \zeta(x)
    \end{array}
  \right)\\
  &=&
  \left(
    \begin{array}{c}
      \xi(x) \\
      -\zeta(x)
    \end{array}
  \right)\\
  &=&
  \left(
    \begin{array}{c}
      \xi(x) \\
      0
    \end{array}
  \right) - 
  \left(
    \begin{array}{c}
      0 \\
      \zeta(x)
    \end{array}
  \right)
\end{eqnarray}
ここで、$I_2$が$2\times2$成分の行列であることに留意しておき、スピノルとして計4成分のままであることをリマインドしておこう。
ゆえに$\psi(x)$の上2成分は$\gamma^5_W$の固有状態であり、その固有値は+1、下2成分の固有値は-1である。
\begin{eqnarray}
  \gamma^5 
  \left(
    \begin{array}{c}
      \xi(x) \\
      0
    \end{array}
  \right)
  &=& 
  +1
  \left(
    \begin{array}{c}
      \xi(x) \\
      0
    \end{array}
  \right)
  \equiv
  \psi_R \\
  \gamma^5 
  \left(
    \begin{array}{c}
      0 \\
      \zeta(x) 
    \end{array}
  \right)
  &=& 
  -1
  \left(
    \begin{array}{c}
      0 \\
      \zeta(x)
    \end{array}
  \right)
  \equiv
  -
  \psi_L
\end{eqnarray}
$
  \left(
    \begin{array}{c}
      \xi(x) \\
     0 
    \end{array}
  \right)
$
、
$
  \left(
    \begin{array}{c}
      0 \\
      \zeta(x)
    \end{array}
  \right)
$
はカイラルスピノル(ワイルスピノル)と呼ばれ、$\psi_R$(スピン右巻き状態、right-handed)、 $\psi_L$(スピン左巻き状態、left-handed)と表される。
また、$\gamma^5$の固有値をカイラリティと呼び、以上の議論で見たように右巻き粒子は+1、左巻き粒子は-1のカイラリティを持つ。
繰り返しになるが以上の議論をまとめると、
\begin{eqnarray}
  \gamma^5\psi = \psi_R - \psi_L \\
  \gamma^5\psi_R = (+1)\psi_R \\
  \gamma^5\psi_L = (-1)\psi_L 
\end{eqnarray}
である。$\gamma_D^5$の固有状態をカイラルスピノル、固有状態をカイラリティと呼ぶ。

ここで、4成分スピノルから右巻き・左巻き状態を取り出すための射影演算子を定義する。
\begin{eqnarray}
  P_R \equiv \frac{1}{2}(I_4+\gamma^5) \\
  P_L \equiv \frac{1}{2}(I_4-\gamma^5)
\end{eqnarray}
実際にこれらの演算子をスピノル$\psi$に作用させると
\begin{eqnarray}
  P_R\psi 
  &=& \frac{1}{2}(I_4+\gamma^5)(\psi_R+\psi_L) \\
  &=& \frac{1}{2}(\psi_R + \psi_L +\gamma^5\psi_R+ \gamma^5\psi_L) \\
  &=& \frac{1}{2}(\psi_R + \psi_L +\psi_R - \psi_L) \\
  &=& \psi_R
\end{eqnarray}
と右巻き成分だけを取り出すことが出来る(左巻きに対しても全く同じ議論で証明できる)\footnote{ここでのキモは$\psi=\psi_R + \psi_L$と単純に線形結合で表されるということを覚えているかどうかだ。あとは、カイラリティの正負に気をつければ簡単に証明することができる}。

\section{カイラル固有状態、ヘリシティ固有状態}

カイラルスピノルの性質を調査するために、電磁場中のディラック方程式を再度用いる。
\begin{eqnarray}
  \left( i\gamma^\mu(\partial_\mu + iqA_\mu) -m \right)\psi = 0
\end{eqnarray}
両辺左から射影演算子$P_R$を作用させると
\begin{eqnarray}
  \frac{1}{2}(I_4+\gamma^5)\left( i\gamma^\mu(\partial_\mu + iqA_\mu) -m \right)\psi = 0
\end{eqnarray}

\section{パイオン崩壊}
カイラリティとヘリシティの議論によく登場するのが、パイ中間子の崩壊である。
$pi^{+}$は$W^+$を介して$pi^+\to \nu_\ell + \ell^+$に崩壊し、荷電レプトンの質量差からも崩壊率は$\Gamma(pi^+\to \nu_e + e^+)>\Gamma(pi^+\to \nu_\mu + \mu^+)$であると予想される。
しかし実際には

\begin{equation}
  \frac{\Gamma(pi^+\to \nu_e + e^+)}{\Gamma(pi^+\to \nu_\mu + \mu^+)} = 1.23\times 10^{-4}
\end{equation}

であり、優位にミューオンへの崩壊のほうが多い実験結果が示されている。
これを説明するのがカイラリティ、ヘリシティである。なぜ電子ではなミューオンへ崩壊することになるのか、$\pi^+$中間子の静止系で考えると説明がつく。
\begin{itemize}
  \item $\pi^+$のスピンは0であるため、$\nu_\ell$、$\ell^+$のスピンは反対方向を向いている(互いに外側、もしくは互いに内側)
  \item SMでは$\nu_\ell$は左巻きしか存在しないので、ニュートリノ側のスピン方向が決まる(内側を向いている)
  \item そのため$\ell^+$のスピンの向きが決定する(内側)
  \item $\ell^+$は運動量方向とスピン方向が逆向きの左巻き粒子ということになるが、弱い相互作用はLH粒子、もしくはRH反粒子にしか結合しない
  \item そのため、LH反粒子とは本来結合しないはずである
  \item $\ell^+$が右巻きになるには、追い越せる速度である必要がある(電子$<<$ミューオンの質量なので、ミューオンが選択される)
\end{itemize}


\chapter{ゲージ理論}


\section{QED}

ゲージ場$A_{\mu}$に対する運動項(kinetic term) は
\begin{equation}
  -\frac{1}{4}F_{\mu\nu}F^{\mu\nu}
\end{equation}
であり、ここで$F_{\mu\nu}$は場の強さ(field strength)を表している。
ゲージ変換の下で:
\begin{eqnarray}
  F^{\mu\nu\prime}
  &=& \partial^{\mu}(A^{\nu}+\frac{1}{q}\partial^{\nu}) - \partial^{\nu}(A^{\mu}+\frac{1}{q}\partial^{\mu}) \\
  &=& \partial^{\mu}A^{\nu} - \partial^{\nu}A^{\mu} = F^{\mu\nu}.
\end{eqnarray}
であり、ゲージ不変性\footnote{ゲージ変換の下で式の形(物理法則)が変わらないということ}を持つ。

ゲージ場$A_{\mu}$の質量項は
\begin{equation}
  \frac{1}{2}m_A^2A_{\mu}A^{\mu},
\end{equation}
と表されるが、これはゲージ変換の下で不変ではない。
ゲージ原理を第一原理として理論体系を構築しようとしているので、この項を入れることはできない。
そのためゲージ場の質量$m_A$は正確に0でなければならない\footnote{ここでの質量$m$は「ゲージ場」の質量であり、ディラック粒子の質量ではない。}。
以上を踏まえて、QEDのラグランジアンは以下の式で表される。
\begin{equation}
  \mathcal{L} = \bar{\psi}(i\gamma^{\mu}\partial_{\mu}-m)\psi + e\bar{\psi}\gamma^{\mu}\psi A_{\mu} -\frac{1}{4}F_{\mu\nu}F^{\mu\nu}.
\end{equation}

\section{QCD}



\chapter{レプトクォーク}

\section{Bファクトリー実験結果}

\subsection{$R(D^*)$}

B中間子のセミレプトニック崩壊に関して、それぞれのレプトンへの分岐比を測定した。
\begin{equation}
  R(D*)=\frac{Br(\bar{B}\to D^{(*)} \tau\nu_\tau)}{Br(B\to D^{(*)} \ell\nu_\tau)},~\mathrm{where}~\ell=e,\mu
\end{equation}
この値が標準模型の理論予測値から$4\sigma$ずれていることが報告されている。
実験結果は、「標準模型では説明できない、選択的に第3世代に強く結合する機構」を示唆しており、
ここからレプトクォークモデル探索が一躍脚光を浴びることとなった。

\begin{figure}
  \centering
  \includegraphics[width=0.3\textwidth]{figure/leptoquark/b_to_ctaunu.pdf}
\end{figure}

\subsection{実験原理}
\subsubsection{Belle実験}

電子・陽電子コライダー実験であり、Bファクトリ実験の一つ。
B中間子分岐比の比を取ることで、分子分母に共通する系統誤差をキャンセルすることができる。


\section{モチベーション}

標準模型を繰り込み可能にするには、次の条件が成り立っている必要がある
\begin{equation}
  \mathrm{Tr} T^a\{T^b,T^c\}
\end{equation}
この条件は、全てのフェルミオンの和を取ることで成り立つのだが、標準模型の範疇ではなぜフェルミオンをひとまとめに
扱うのかは明確に示されていない。
確かに同じ世代数、同じ電弱相互作用を受けるのだが、あくまで理論体系は別個のものである。
そのため、レプトンとクォークはどこかで直接の関係を持っているとする理論が多数提唱されている。
レプトクォークも、その観点で提唱されているモデルである。


\section{疑問}
\begin{itemize}
    \item SMにおける量子異常とは?
    \item どんなレプトクォークがあれば、Bアノマリーが説明づくのか?
    \item 質量下限値、上限値は理論から制限はあるか?
    \item SMでなぜFCNCは禁止されている?
\end{itemize}




\chapter{レプトクォーク}

\section{Bファクトリー実験結果}

\subsection{$R(D^*)$}

B中間子のセミレプトニック崩壊に関して、それぞれのレプトンへの分岐比を測定した。
\begin{equation}
  R(D*)=\frac{Br(\bar{B}\to D^{(*)} \tau\nu_\tau)}{Br(B\to D^{(*)} \ell\nu_\tau)},~\mathrm{where}~\ell=e,\mu
\end{equation}
この値が標準模型の理論予測値から$4\sigma$ずれていることが報告されている。
実験結果は、「標準模型では説明できない、選択的に第3世代に強く結合する機構」を示唆しており、
ここからレプトクォークモデル探索が一躍脚光を浴びることとなった。

\begin{figure}
  \centering
  \includegraphics[width=0.3\textwidth]{figure/leptoquark/b_to_ctaunu.pdf}
\end{figure}

\subsection{実験原理}
\subsubsection{Belle実験}

電子・陽電子コライダー実験であり、Bファクトリ実験の一つ。
B中間子分岐比の比を取ることで、分子分母に共通する系統誤差をキャンセルすることができる。


\section{モチベーション}

標準模型を繰り込み可能にするには、次の条件が成り立っている必要がある
\begin{equation}
  \mathrm{Tr} T^a\{T^b,T^c\}
\end{equation}
この条件は、全てのフェルミオンの和を取ることで成り立つのだが、標準模型の範疇ではなぜフェルミオンをひとまとめに
扱うのかは明確に示されていない。
確かに同じ世代数、同じ電弱相互作用を受けるのだが、あくまで理論体系は別個のものである。
そのため、レプトンとクォークはどこかで直接の関係を持っているとする理論が多数提唱されている。
レプトクォークも、その観点で提唱されているモデルである。


\section{疑問}
\begin{itemize}
    \item SMにおける量子異常とは?
    \item どんなレプトクォークがあれば、Bアノマリーが説明づくのか?
    \item 質量下限値、上限値は理論から制限はあるか?
    \item SMでなぜFCNCは禁止されている?
\end{itemize}



\appendix

\chapter{Appendix}
\section{$\gamma$行列の性質}
反交換関係:
\begin{eqnarray}
  \{ \gamma^\mu, \gamma^\nu \} = \gamma^\mu\gamma^\nu + \gamma^\nu\gamma^\mu = 2\eta^{\mu\nu}
\end{eqnarray}
ここから、$\mu=\nu$のときに$(+1,-1,-1,-1)$のいずれかの値を取り(計量テンソルの対角成分)、$\mu\neq\nu$の場合には0の値を取ることがわかる(非対角成分)。

$\mu=\nu$の場合(例えば$\mu=\nu=0$)、
\begin{eqnarray}
  \gamma^0\gamma^0 + \gamma^0\gamma^0 &=& 2\eta^{00}\\
  2\gamma^0\gamma^0  &=& 2 \\
  \gamma^0\gamma^0   &=& +1 
\end{eqnarray}
また0以外の値の場合は計量テンソルの定義から、右辺が-1になる。

また$\mu \neq \nu$の場合、
\begin{eqnarray}
  \gamma^\mu\gamma^\nu + \gamma^\nu\gamma^\mu &=& 0 \\
  \gamma^\mu\gamma^\nu &=& (-1)\gamma^\nu\gamma^\mu 
\end{eqnarray}
が成り立ち、行列の積の順序を入れ替えるとその回数分だけ$(-1)$が掛かる。
つまりガンマ行列は限定的に(符号が付くという条件があるだけで)入れ替え可能で、符号にだけ注意しておけばよい。


\begin{eqnarray}
  \gamma^5 = i\gamma^0\gamma^1\gamma^2\gamma^3
\end{eqnarray}

\begin{eqnarray}
  (\gamma^5)^2 
  &=& (i\gamma^0\gamma^1\gamma^2\gamma^3)^2 \\
  &=& (i\gamma^0\gamma^1\gamma^2\gamma^3)(i\gamma^0\gamma^1\gamma^2\gamma^3) \\
  &=& (-1)\gamma^0\gamma^1\gamma^2\gamma^3\gamma^0\gamma^1\gamma^2\gamma^3 \\
  &=& (-1)(-1)^3\gamma^0\gamma^0\gamma^1\gamma^2\gamma^3\gamma^1\gamma^2\gamma^3 \\
  &=& (-1)(-1)^3\gamma^0\gamma^0(-1)^2\gamma^1\gamma^1\gamma^2\gamma^3\gamma^2\gamma^3 \\
  &=& (-1)(-1)^3\gamma^0\gamma^0(-1)^2\gamma^1\gamma^1(-1)\gamma^2\gamma^2\gamma^3\gamma^3 \\
  &=& (-1)(-1)^3(-1)^2(-1) \\
  &=& -1
\end{eqnarray}


\section{湯川相互作用}
スカラー場(擬スカラー場)とフェルミオンの相互作用を表す。
パイオンによる核力を記述する際に用いられることと、標準理論の中ではスカラーボソンであるヒッグス場とフェルミオン場の相互作用を記述する際に用いられる。
スカラー場の場合(ヒッグス場の場合)
\begin{eqnarray}
  \mathcal{L} = -g\bar{\psi}\phi\psi
\end{eqnarray}
ここで$g$は湯川結合定数を表す。また擬スカラー場の場合
\begin{eqnarray}
  \mathcal{L} = -g\bar{\psi}i\gamma^5\phi\psi
\end{eqnarray}
となる。これは双一次形式で見た形から連想できる。
標準理論を扱う場合にはスカラー場のパターンが最も重要である。



\end{document}
