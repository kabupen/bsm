\chapter{ゲージ理論}


\section{QED}

ゲージ場$A_{\mu}$に対する運動項(kinetic term) は
\begin{equation}
  -\frac{1}{4}F_{\mu\nu}F^{\mu\nu}
\end{equation}
であり、ここで$F_{\mu\nu}$は場の強さ(field strength)を表している。
ゲージ変換の下で:
\begin{eqnarray}
  F^{\mu\nu\prime}
  &=& \partial^{\mu}(A^{\nu}+\frac{1}{q}\partial^{\nu}) - \partial^{\nu}(A^{\mu}+\frac{1}{q}\partial^{\mu}) \\
  &=& \partial^{\mu}A^{\nu} - \partial^{\nu}A^{\mu} = F^{\mu\nu}.
\end{eqnarray}
であり、ゲージ不変性\footnote{ゲージ変換の下で式の形(物理法則)が変わらないということ}を持つ。

ゲージ場$A_{\mu}$の質量項は
\begin{equation}
  \frac{1}{2}m_A^2A_{\mu}A^{\mu},
\end{equation}
と表されるが、これはゲージ変換の下で不変ではない。
ゲージ原理を第一原理として理論体系を構築しようとしているので、この項を入れることはできない。
そのためゲージ場の質量$m_A$は正確に0でなければならない\footnote{ここでの質量$m$は「ゲージ場」の質量であり、ディラック粒子の質量ではない。}。
以上を踏まえて、QEDのラグランジアンは以下の式で表される。
\begin{equation}
  \mathcal{L} = \bar{\psi}(i\gamma^{\mu}\partial_{\mu}-m)\psi + e\bar{\psi}\gamma^{\mu}\psi A_{\mu} -\frac{1}{4}F_{\mu\nu}F^{\mu\nu}.
\end{equation}

\section{QCD}
