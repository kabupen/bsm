
\chapter{レプトクォーク}

\section{Bファクトリー実験結果}

\subsection{$R(D^*)$}

B中間子のセミレプトニック崩壊に関して、それぞれのレプトンへの分岐比を測定した。
\begin{equation}
  R(D*)=\frac{Br(\bar{B}\to D^{(*)} \tau\nu_\tau)}{Br(B\to D^{(*)} \ell\nu_\tau)},~\mathrm{where}~\ell=e,\mu
\end{equation}
この値が標準模型の理論予測値から$4\sigma$ずれていることが報告されている。
実験結果は、「標準模型では説明できない、選択的に第3世代に強く結合する機構」を示唆しており、
ここからレプトクォークモデル探索が一躍脚光を浴びることとなった。

\begin{figure}
  \centering
  \includegraphics[width=0.3\textwidth]{figure/leptoquark/b_to_ctaunu.pdf}
\end{figure}

\subsection{実験原理}
\subsubsection{Belle実験}

電子・陽電子コライダー実験であり、Bファクトリ実験の一つ。
B中間子分岐比の比を取ることで、分子分母に共通する系統誤差をキャンセルすることができる。


\section{モチベーション}

標準模型を繰り込み可能にするには、次の条件が成り立っている必要がある
\begin{equation}
  \mathrm{Tr} T^a\{T^b,T^c\}
\end{equation}
この条件は、全てのフェルミオンの和を取ることで成り立つのだが、標準模型の範疇ではなぜフェルミオンをひとまとめに
扱うのかは明確に示されていない。
確かに同じ世代数、同じ電弱相互作用を受けるのだが、あくまで理論体系は別個のものである。
そのため、レプトンとクォークはどこかで直接の関係を持っているとする理論が多数提唱されている。
レプトクォークも、その観点で提唱されているモデルである。


\section{疑問}
\begin{itemize}
    \item SMにおける量子異常とは?
    \item どんなレプトクォークがあれば、Bアノマリーが説明づくのか?
    \item 質量下限値、上限値は理論から制限はあるか?
    \item SMでなぜFCNCは禁止されている?
\end{itemize}

