\chapter{量子力学}
\section{基本的な関係式一覧}

\subsubsection{転置}
\begin{equation}
  (H_{ij})^{T}=(H_{ji})
\end{equation}
対角成分は変換を受けず、非対角成分のみが変換を受ける。

\begin{equation}
  \left(
    \begin{array}{cccc}
      a_{11} & a_{12} & a_{13} & a_{14} \\
      a_{21} & a_{22} & a_{23} & a_{24} \\
      a_{31} & a_{32} & a_{33} & a_{34} \\
      a_{41} & a_{42} & a_{43} & a_{44} 
    \end{array}
  \right)
    ^{T}
    =
  \left(
    \begin{array}{cccc}
      a_{11} & a_{21} & a_{31} & a_{41} \\
      a_{12} & a_{22} & a_{32} & a_{42} \\
      a_{13} & a_{32} & a_{33} & a_{43} \\
      a_{14} & a_{24} & a_{34} & a_{44} 
    \end{array}
  \right)
\end{equation}

\subsubsection{エルミート共役}
「エルミート共役をとる」とは、(1)行列を転置して、(2)複素共役をとったもの。
\begin{equation}
  (H_{ij})^{\dagger}=(H_{ji})^{*}
\end{equation}


\subsubsection{エルミート演算子、エルミート行列}
エルミート共役をとった演算子が自分自身である演算子のこと。
「この行列はエルミートである」とも表現される
\begin{equation}
H^{\dagger}=H
\end{equation}

\subsubsection{ユニタリー演算子、ユニタリー行列}
エルミート共役をとったものと自分自身の積が可換で、かつ単位行列に等しい演算子。
\begin{equation}
H^{\dagger}H=HH^{\dagger}=I
\end{equation}

\subsubsection{パウリ行列}
\begin{equation}
  \sigma^1 = 
  \left(
    \begin{array}{cc}
      0 & 1 \\
      1 & 0 
    \end{array}
  \right),~
  \sigma^2 = 
  \left(
    \begin{array}{cc}
      0 & -i \\
      i & 0 
    \end{array}
  \right),~
  \sigma^3 = 
  \left(
    \begin{array}{cc}
      1 & 0 \\
      0 & -1 
    \end{array}
  \right)
\end{equation}

\section{基本的な関係式一覧(量子力学)}
\begin{eqnarray}
  E &\to& i\frac{\partial}{\partial t} \\
  p &\to& -i\nabla \\
  p^{\mu} &\to& i\partial^{\mu} = i(\partial^0, -\partial^1, -\partial^2, -\partial^3)
\end{eqnarray}



